\section*{General Abstract}
We do not want valuable technology we send out in space to be destroyed by the population of rocks surrounding Earth. Therefore, it is important to know exactly how many rocks are out there and how big they are. Our portable video camera device can detect and then record the rocks burning up in Earth's atmosphere. This project focuses on writing software to see how bright those rocks are based off of the video collected by the device. The brightnesses can be used later to estimate their sizes.

\section*{Technical Abstract}
When designing new satellites, incorporating damage mitigation techniques is currently difficult due to a lack of understanding about the near-Earth small asteroid population. In an effort to investigate this population, we have constructed a portable all-sky camera to continuously monitor the night sky. It has the capability to detect meteors ablating in Earth's atmosphere and record their brightness as pixel values. To begin to measure the number and size distribution of the meteoroids that later became these meteors, this project focuses on writing software to automatically measure a meteor's brightness, and  calibrate it to a known astronomic luminosity, which can later be used to estimate its size. When ran on videos from NASA, The program produced similar light curves to theirs, showing a strong proof of concept.
