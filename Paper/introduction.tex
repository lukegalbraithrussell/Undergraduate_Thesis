\chapter{Introduction}

Outer space, while mostly space, is not entirely empty.  There are planets, stars, comets, and many other objects, but here we are concerned about meteors. Specifically, we are interested in the smallest meteors. While rocks the size of pebbles may not seem dangerous to those of us protected by Earth's atmosphere, they pose a great threat to equipment and life beyond that atmosphere. We desire to know how many of these objects are in orbit near Earth in order to estimate the necessary level of protection for our structures and to better estimate the lifespans of those structures.

Currently, one of the ways meteors can be detected entering our atmosphere is by all-sky cameras detecting the bright streaks they leave as they ablate in the atmosphere. This brightness is what gives them another name: fireballs. All-sky cameras are set up by NASA and other groups\cite{Jenniskens2011,Trigo-Rodriguez2007}. They are installed on roofs or mounts and hooked up to desktop computers in buildings. It is a big and inflexible setup. This infrastructure limits the amount of setups that currently exist, which in turn limits the amount of total sky coverage we can get. Our all-sky camera explores a new design: one in which its detection software is housed inside its small, basketball-sized chassis. It currently depends on external power, but low power consumption makes it a strong candidate to run off battery power if needed. We aim to show that our tiny unit (named D6\footnote{D6 is named after a droid in the Star Wars universe similar to R2-D2, but is green and owned by Rebellion hero Wedge Antilles.}) is capable of observing and analyzing fireballs just as reliably as the larger, more established networks.

We are using our all-sky camera to detect meteors, implementing photometry to calculate the meteors' magnitudes, and then extrapolating their masses from those magnitudes. There have been many other studies by larger all-sky networks to create a model for the population distribution. We have no reason to expect our data to deviate greatly from this trend. In fact, we are hoping to test the effectiveness of our all-sky camera setup by comparing its findings to the established model. Also, while the general trend is well known, there is no guarantee that it is as precise as it could be. We are curious about any possible small discrepancies within the model regarding the smaller size meteoroids, and believe that if any discrepancies are there, they would only become apparent if data was being collected much more frequently than the current rate.

Chapter 2 details the information needed to understand what fireballs are and why they are important. The size of a meteor is determined through a long chain of physical relations that begin at its apparent magnitude. The photometric analysis done to find that apparent magnitude is explained in detail in Chapter 3. The actual results are then presented in Chapter 4.

